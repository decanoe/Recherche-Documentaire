\documentclass{article}
\usepackage{mathtools, stmaryrd}
\usepackage{xparse}
\usepackage{amsmath}
\usepackage{cases}
\usepackage{framed}

\title{Moteur d'inférence}
\author{
    Noé De Caestecker
\and
    Killian Darras
}
\date{\today}

\begin{document}
\maketitle
\tableofcontents

\begin{abstract}
    Le but du projet est de créer un système de recherche documentaire permettant des recherches plus ou moins avancées sur un corpus de document fourni.
\end{abstract}

% #region ================ Système expert ===================
\section{Choix d'implémentation}
Le projet a été réalisé intégralement en python, en utilisant le moins de librairies externes possibles (librairie graphique et gestion de fichier seulement)
afin de mettre en œuvre le plus de techniques par nous même.

Le système de recherche permet l'utilisation d'un caractère joker $*$ et fourni les réponses en les ordonnant dans un ordre décroissant de pertinence.
% #endregion

% #region ================ Index ===================
\section{Index}
L'index suit une structure d'arbre 2-4 où chaque mot est stocké ainsi que ses rotations (pour l'utilisation du caractère joker).
Chaque feuille de l'index est une structure qui représente un mot et stocke l'ensemble des poids de ce mot pour chaque document qui le contient au moins 1 fois.

Afin d'éviter les redondances dans les structures de poids, chaque rotation d'un mot ne stocke que le mot dont il provient au lieu de ses poids. Si une recherche tombe
sur l'une de ces rotations, une seconde recherche dans l'arbre est réalisée pour trouver la bonne liste de poids. Cette implémentation rallonge le temps
d'exploration, mais permet de réduire grandement la taille de l'index (la taille est divisé par 4 en moyenne). De plus, le temps d'exploration est négligeable
pour des parcours sans caractère joker (ce qui est le cas pour ce parcours supplémentaire).
% #endregion

% #region ================ Pertinance ===================
\section{Calcul de la pertinence}

% #endregion
\pagebreak

\end{document}