\documentclass{article}
\usepackage{mathtools, stmaryrd}
\usepackage{xparse}
\usepackage{amsmath}
\usepackage{cases}
\usepackage{framed}

% #region ================ newcommands ===================
\newcommand{\shellcmd}[1]{\\\indent\indent\texttt{\footnotesize\# #1}}
% #endregion

\title{Moteur d'inférence}
\author{
    Noé De Caestecker
\and
    Tino Dolbeau
}
\date{\today}

\begin{document}
\maketitle
\tableofcontents

\begin{abstract}
    L'objectif de ce projet est de programmer un moteur d'inférence d'ordre Zéro+ au minimum adapté à un système expert de notre choix.
    Ce moteur d'inférence doit pouvoir être utilisé pour d'autres applications similaires à celle pour laquelle il a été créé. 
\end{abstract}

% #region ================ Système expert ===================
\section{Système expert}
Afin de disposer rapidement d'une base pour commencer l'implémentation du moteur d'inférence, nous avons décidé de nous orienter
vers une application de classement d'espèces animales. L'application prendra une description complète ou non des caractéristiques d'un animal
(nombre de membres, squelette, poils, \dots) et en déduira l'espèce ou la famille de l'animal (gastéropode, insecte, arachnide, \dots).
% #endregion

\pagebreak

% #region ================ Syntaxe ===================
\section{Syntaxe}
La syntaxe choisie se divise en 2 concepts : les faits et les règles.

Les faits seront composés d'une variable (sous forme d'une chaine de caractères ne contenant que des lettres et des underscores),
d'un symbole arithmétique ou d'affectation et d'une valeur (valeur numérique ou chaine de caractères). Par exemple :
\begin{itemize}
    \item \makebox[5cm][l]{nom\_de\_variable} pour une variable booléenne
    \item \makebox[5cm][l]{-nom\_de\_variable} pour la négation d'une variable booléenne
    \item \makebox[5cm][l]{nom\_de\_variable = 5.3} pour une variable numérique
    \item \makebox[5cm][l]{nom\_de\_variable != 1.68} pour une inégalité de variable numérique
    \item \makebox[5cm][l]{nom\_de\_variable $>$ 0.3} pour une comparaison de variable numérique
    \item \makebox[5cm][l]{nom\_de\_variable = bleu} pour une variable textuelle
    \item \makebox[5cm][l]{nom\_de\_variable != rouge} pour une inégalité de variable textuelle
\end{itemize}

Les règles seront quant à elles composés d'un nom de règle, d'une liste de faits représentant les prémisses et d'une autre représentant les conséquents. Par exemple :
\begin{itemize}
    \item R1: A =\> B \\\makebox[1cm][l]{} une règle avec 1 prémisse booléenne et 1 conséquent booléen
    \item R2: A, B = 3, H =\> B $>$ 2, D $<$ 2 \\\makebox[1cm][l]{} une règle avec 3 prémisses et 2 conséquents
\end{itemize}

Une base de faits et une base de règles d'exemple sont disponibles dans le dossier $bases$ du répertoire du projet.
% #endregion

\pagebreak

% #region ================ Utilisation du projet ===================
\section{Utilisation}
\subsection{Compilation}
Pour compiler le projet, exécuter les commandes :
\shellcmd{cd ./build}
\shellcmd{cmake ..}
\shellcmd{make}
L'exécutable sera ensuite accessible depuis le dossier build avec la commande
\shellcmd{./moteur\_inference <options>}

\subsection{Exécution \& options}
Une fois le projet compilé, et pour pouvoir l'exécuter, une base de règle et une base de faits initiaux doivent être fournis.
Les deux premiers arguments de l'exécutable représentent les chemins vers ces deux bases.

Par exemple, l'utilisation des bases fournies avec le projet donne la commande
\shellcmd{./moteur\_inference ../bases/rules.txt ../bases/facts.txt <options>}\\

L'exécutable attend plusieurs autres options qui seront discutés dans la suite du document. Entre autres, la stratégie à utiliser doit être renseigné
en utilisant les options -fc ou -bc respectivement pour le chainage avant et arrière (\textit{forward chaining} et \textit{backward chaining})
\\

Une liste des options est accessible via la commande
\shellcmd{./moteur\_inference -help}

ou
\shellcmd{./moteur\_inference -h}
% #endregion

\pagebreak
\section{Stratégies}
% #region ================ chaînage avant ===================
\subsection{Chaînage avant}
Le chainage avant (utilisable avec l'option -f ou -fc) sature la base de faits en utilisant la base de règles.
Un but peut être donné en utilisant l'option -b suivie d'un fait objectif (entre guillemets).
Si un but est donné, les règles sont appliquées jusqu'a ce que le but soit atteint ou que la base de faits soit saturée.

Lors du chainage avant, l'ensemble des règles utilisés sont indiqués les une après les autres.
La base de fait finale est ensuite affichée avec une indication si le fait but a été atteint ou non.
\\

Exemple de sortie pour la commande
\shellcmd{./moteur\_inference ../bases/rules.txt ../bases/facts.txt -fc}
\vspace{-2ex}
\begin{framed}
\vspace{-2ex}
\begin{verbatim}
- R1: tete => animal  
- R4: animal, squelette_interne, crane => vertebre  
- R5: vertebre, squelette_os => osteichtyen  
- R6: osteichtyen, membre = 4 => tetrapode  
- R7: tetrapode, gesier => oiseau_ou_croco  
- R9: oiseau_ou_croco, trou_os_tempe => crocodilien  
6 rules applied
Fact base saturated
Final fact base: 
[
        tete
        bouche
        yeux
        squelette_interne
        crane
        squelette_os
        membre = 4
        gesier
        trou_os_tempe
        animal
        vertebre
        osteichtyen
        tetrapode
        oiseau_ou_croco
        crocodilien
]
\end{verbatim}
\vspace{-2ex}
\end{framed}
% #endregion
\pagebreak
% #region ================ chaînage arrière ===================
\subsection{Chaînage arrière}
Le chainage arrière (utilisable avec l'option -b ou -bc) donne une démonstration d'un fait donné en paramètre.
Contrairement au chainage arrière, le but doit être donné en utilisant l'option -b suivie du fait objectif (entre guillemets).

La sortie du chainage arrière est un arbre des règles à utiliser pour démontrer le fait voulu. Si un fait demandable nécessaire
n'est pas présent dans la base de faits et que sa négation n'y est pas non plus, le programme peut alors demander à l'utilisateur si ce fait est
vrai, faux ou indéterminé.
\\

Exemple de sortie pour la commande
\shellcmd{./moteur\_inference ../bases/rules.txt ../bases/facts.txt -bc -g ``crocodilien''}
\vspace{-4ex}
\begin{framed}
\vspace{-2ex}
\begin{verbatim}
R9: oiseau_ou_croco, trou_os_tempe => crocodilien  
 +- R7: tetrapode, gesier => oiseau_ou_croco  
 |  +- R6: osteichtyen, membre = 4 => tetrapode  
 |  |  +- R5: vertebre, squelette_os => osteichtyen  
 |  |  |  +- R4: animal, squelette_interne, crane => vertebre  
 |  |  |  |  +- R1: tete => animal  
 |  |  |  |  |  +- tete
 |  |  |  |  +- squelette_interne
 |  |  |  |  \- crane
 |  |  |  \- squelette_os
 |  |  \- membre = 4
 |  \- gesier
 \- trou_os_tempe
\end{verbatim}
\vspace{-2ex}
\end{framed}

Exemple de demande de fait pour "demandable $>$ 2"
\vspace{-2ex}
\begin{framed}
\vspace{-2ex}
\begin{verbatim}
Is this true ? (y/n/?) demandable > 2
_
\end{verbatim}
\vspace{-2ex}
\end{framed}
% #endregion

\pagebreak
\section{Critères de choix}
En plus de choisir la stratégie à utiliser, il est possible de préciser selon quel critère les règles sont choisies.
Par défaut, les règles sont testées dans l'ordre de leur présence dans la base de règles, ce qui n'est en général pas le meilleur choix.
Deux critères sont possibles : le nombre de prémisses ou les prémisses les plus récemment ajoutées dans la base de faits.
% #region ================ Max Premisses ===================
\subsection{Nombre de prémisses}
Le critère du nombre de prémisses peut être activé avec l'option -max\_premisses. Si c'est le cas, les règles ayant le plus grand nombre de prémisses
seront testés en premier. Si plusieurs règles peuvent être appliquées et ont le même nombre de prémisses, la première dans l'ordre de la base de règles
sera sélectionnée.
% #endregion
% #region ================ recent_premisses ===================
\subsection{Nombre de prémisses}
Le critère des prémisses les plus récentes peut être activé avec l'option -recent\_premisses. Les règles utilisant les faits les plus récemment ajoutés dans
la base de faits seront testés en premier. Tout comme le critère du nombre de prémisses, en cas d'égalité, la première règle concernée dans l'ordre de la
base de règles sera sélectionnée.

Le calcul utilisé pour le choix de la règle est la somme des indices de chaque fait dans la base de faits (les faits les plus récents ayant
les indices les plus grands).
% #endregion

\pagebreak
\section{Contraintes de l'implémentation sur le langage}


\end{document}